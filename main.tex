\documentclass{beamer}
\usetheme{Madrid}
\usecolortheme{default}
\usepackage[]{minted}
\usepackage{fancyvrb}
\usepackage{tikz}
\usepackage{multicol}
\usepackage{graphicx}
\usepackage{hyperref}
\usepackage{subfigure}
\usetikzlibrary{positioning,matrix,automata,angles,quotes,positioning,fit,backgrounds,intersections,arrows,calc,tikzmark}

\setbeamercovered{transparent}

\title[Linux on a floppy]
{Can we boot Linux from just a floppy?}
\subtitle{An exercise in minimalism.}

\author[Kamila Szewczyk] % (optional, for multiple authors)
{Kamila Szewczyk, \textit{Association for Computational Heresy}}

\date[19 Sesja Linuksowa] % (optional)
{19 Sesja Linuksowa, Wroclaw, April 2025}

\definecolor{uoftblue}{RGB}{6,41,88}
\setbeamercolor{titlelike}{bg=uoftblue}
\setbeamerfont{title}{series=\bfseries}

\setcounter{tocdepth}{4}

\begin{document}

\frame{\titlepage}

\section{Introduction}
\begin{frame}[t]
\frametitle{Introduction}
\begin{itemize}
  \item Some common storage media.
\end{itemize}
\begin{figure}
  \centering
  \subfigure[CD-ROM]{
    \includegraphics[width=0.2\textwidth]{cd.jpg}
  }
  \hspace{0.5cm}
  \subfigure[USB drive]{
    \includegraphics[width=0.2\textwidth]{thumb-drive.jpg}
  }
  \hspace{0.5cm}
  \subfigure[SD Card]{
    \includegraphics[width=0.2\textwidth]{sd.jpg}
  }
\end{figure}
\begin{itemize}
  \item Their capacities are usually overwhelming and measured in gigabytes ($10^9$ bytes).
  \item All of them constitute popular boot and installation media for Linux.
  \item Do we really need this much storage just to boot an operating system?
\end{itemize}
\end{frame}

\begin{frame}[t]
\frametitle{Introduction}
\begin{itemize}
  \item Prior art:
  \begin{itemize}
    \item Damn Small Linux (DSL) - 50MB - A live CD distribution for 486+ CPUs, 8MB of RAM.
    \item Tiny Core Linux - 17MB - Perhaps the most popular minimalist distribution.
    \item NanoLinux - 14MB - Discontinued project.
  \end{itemize}
  \item All of those contenders are quite small and still provide many interesting features like web browsers, text editors or even games.
  \item They also lag behind in kernel versions.
  \item What if we didn't want \textit{any} of the nice features, and simply wanted a minimal TTY environment that still formally runs Linux?
\end{itemize}
\end{frame}

\begin{frame}[t]
\frametitle{Introduction}
\begin{itemize}
  \item As a goalpost of our experiment, we will use this magical device:
\end{itemize}
\begin{figure}
  \centering
  \includegraphics[width=0.4\textwidth]{floppy.jpg}
  \caption{A 3.5'' floppy disk.}
\end{figure}
\begin{itemize}
  \item Each of these holds about 1.44MB of data, about 0.2\% of a standard 700MB CD-ROM.
  \item The best widespread contender, Tiny Core Linux, needs at least 10 of these to fit.
\end{itemize}
\end{frame}

\section{First steps}
\begin{frame}[t]
\frametitle{Toolchain}
\begin{itemize}
  \item It is clear that we will have to compile the kernel and the userland ourselves.
  \item Because 32-bit x86 code tends to be smaller than its 64-bit counterpart, we will build a i486+ kernel.
  \begin{itemize} \item This has some more advantages: we turn off fragments of the kernel code responsible for complicated hardware instructions that the 486 simply doesn't have. \end{itemize}
  \item But we don't want just \textit{any} 486 ELF toolchain: $\texttt{glibc}$ is much too bulky for our needs.
  \item Instead, we will use $\texttt{musl}$, a lightweight C library, which tends to be much smaller and compatible with $\texttt{glibc}$ on many counts.
  \item Equipped with $\texttt{i486-linux-musl-gcc}$, we may continue.
  \item In case you want to follow along, you can get this from $\texttt{https://musl.cc/}$.
\end{itemize}
\end{frame}

\begin{frame}[t,fragile]
\frametitle{Toolchain: experiment}
\begin{Verbatim}[commandchars=\\\{\}]
 % \textcolor{red}{cat hello.c}
 #include <stdio.h>
 int main(void) { puts("Hello, world!"); for (;;) ; }
 % \textcolor{red}{i486-linux-musl-gcc hello.c -o hello -Os -static -s}
 % \textcolor{red}{wc -c hello && ./hello}
 4948 hello
 Hello, world!
 % \textcolor{red}{ldd hello}
   not a dynamic executable
\end{Verbatim}
\begin{itemize}
  \item The resulting 32-bit program only uses system calls and does not depend on any shared libraries.
  \item 5KB for a simple program is not fantastic, but it is a good start.
  \item Let's not get off track. We will get back to that later.
\end{itemize}
\end{frame}

\begin{frame}[t,fragile]
\frametitle{Kernel \& initrd}
\begin{itemize}
  \item Let's try to get something booting first.
  \item The specific kernel that we will be working with is \verb|linux-6.13.8| - most recent version at the time of writing.
  \item Compiling the kernel by itself is pretty basic nitty-gritty.
  \begin{itemize}
    \item Use \verb|arch/x86/configs/i386_defconfig| as a base.
    \item Enable TTY and \verb|printk| support (for the latter you need expert settings enabled).
    \item Then: \verb|make ARCH=x86 CROSS_COMPILE=i486-linux-musl- bzImage|.
  \end{itemize}
  \item Before we can compile and boot the kernel, we need to prepare the initrd.
  \item The $\textit{Hello, world!}$ program is a good initial candidate: together with a small bootloader, it will leave us with a lot of space for the kernel.
  \item We put the init program in a CPIO archive, which is a standard format for initrd images.
\end{itemize}
\end{frame}

\begin{frame}[t,fragile]
\frametitle{Kernel \& initrd}
\begin{itemize}
  \item Boot with the following command:
\begin{Verbatim}
  qemu-system-i386 -kernel arch/x86/boot/bzImage \
                   -initrd ./rootfs.cpio.gz
\end{Verbatim}
\end{itemize}
\begin{center}
  \includegraphics[width=0.7\textwidth]{hello.png}
\end{center}
\end{frame}

\begin{frame}[t,fragile]
\frametitle{Kernel \& initrd}
\begin{itemize}
  \item Excellent! We have a working kernel and initrd.
  \item Bad news: the kernel image is 10.8 MiB.
  \item Before we start trying to remove stuff from the kernel, let's try to build a functional initrd using BusyBox. Then it's easier to tell if the stuff we are removing is actually used by something.
  \item Specifically, we will use BusyBox 1.35.0. The binary is quite large, at around 1.0 MiB, but it is statically linked and contains a lot of useful tools.
\end{itemize}
\end{frame}

\begin{frame}[t,fragile]
\frametitle{Kernel \& initrd}
\begin{itemize}
  \item \verb|init| program: install BusyBox, mount \verb|devtmpfs| and run \verb|/bin/init|.
\end{itemize}
\vspace{0.25cm}
\begin{minted}[fontsize=\scriptsize,frame=single]{c}
int main(int argc, char *argv[]) {
  pid_t pid = fork();
  if (pid == 0) {
    char * args[] = { "/bin/busybox", "--install", "-s", NULL };
    execv(args[0], args);
    perror("execv failed"), exit(1);
  } else if (pid < 0)
    perror("fork failed"), exit(1);
  int status; waitpid(pid, &status, 0);
  if (!WIFEXITED(status))
    perror("waitpid"), exit(1);
  if (mount("none", "/dev", "devtmpfs", 0, "")) perror("mount"), exit(1);
  if (mount("none", "/proc", "proc", 0, "")) perror("mount"), exit(1);
  if (mount("none", "/sys", "sysfs", 0, "")) perror("mount"), exit(1);
  // /bin/init is dmesg -n 1;exec sh
  char * args[] = { "/bin/busybox", "sh", "/bin/init", NULL };
  execv(args[0], args);  perror("execv"); exit(1);
}
\end{minted}
\end{frame}

\begin{frame}[t,fragile]
\frametitle{Kernel \& initrd}
\begin{center}
  \includegraphics[width=0.8\textwidth]{busybox1.png}
\end{center}
\end{frame}

\section{Kernel}
\begin{frame}[t,fragile]
\frametitle{Stripping the kernel}
\begin{itemize}
  \item We have a Linux system that fits in about 11 MiB. The CPIO with the \verb|initrd| is about 699 KiB: we will have to look into that, because surely we can improve on it.
  \item The next two steps are compressing these two components well enough to fit in a total of 1.44 MB. We will start with the kernel, as it's the biggest offender.
  \begin{itemize}
    \item First step: Use XZ compression, disable the support for all other codecs except \verb|gzip| for \verb|initrd|, compile with \verb|-Os|.
    \item These changes alone bring the kernel down to about 6.7 MiB.
  \end{itemize}
  \item Unfortunately, together with the initrd we are still quite far away from our goal:
\begin{Verbatim}
  6763008 arch/x86/boot/bzImage
   669206 rootfs.cpio.gz
  7432214 total
\end{Verbatim}
\end{itemize}
\end{frame}

\begin{frame}[t,fragile]
\frametitle{Stripping the kernel}
\begin{itemize}
  \item We will squeeze the kernel as much as possible by removing device drivers, filesystems, networking features, debugging options, and other features that we don't need.
  \item Further, the build script supports embedding the initrd in the kernel image. We will use \verb|xz| compression (keeping in mind that we need \verb|--check=crc32| for the kernel to boot).
  \item The result? \verb|1282560 arch/x86/boot/bzImage| - the kernel + initrd fit in 1.28MB, much below the 1.44MB limit.
\end{itemize}
\end{frame}

\begin{frame}[t,fragile]
\frametitle{Stripping the kernel}
\begin{itemize}
  \item The result? \verb|1282560 arch/x86/boot/bzImage| - the kernel + initrd fit in 1.28MB, much below the 1.44MB limit.
  \item That's very nice, but we can't really do anything with our operating system other than idling in the shell.
  \item I personally don't like \verb|vi| and would prefer a different text editor. Also, I would like to put some cool programs in the initramfs to play with - that would be best achieved with a dynamically linked libc.
  \item Also, we haven't actually figured out how to make this boot off an actual floppy (yet).
\end{itemize}
\end{frame}

\begin{frame}[t,fragile]
\frametitle{Goals}
\begin{itemize}
  \item Now that we have a MVP, we want to accomplish the following:
  \begin{enumerate}
    \item Build a bootable floppy disk instead of relying on \verb|qemu-system-i386 -kernel arch/x86/boot/bzImage| to boot.
    \item Strip down busybox to remove things that we won't need. For example user management, various mkfs/fsck programs for filesystems that we don't support.
    \item Add more programs to the initramfs and features to the kernel: interpreters, compilers, games, etc.
  \end{enumerate}
  \item Of course, eventually we will run out of space. Then we will look into custom compression.
\end{itemize}
\end{frame}

\begin{frame}[t,fragile]
\frametitle{Adding networking}
\begin{itemize}
  \item We will enable networking, E1000 support, the TCP/IP stack and the basic security features.
  \item This unfortunately makes the kernel swell up quite a bit.
  \item Configuring networking with busybox is tricky. I wrote the following script:
\end{itemize}
\begin{minted}[fontsize=\scriptsize,frame=single]{bash}
#!/bin/sh
ip link set eth0 up
LEASE=$(udhcpc -i eth0 -n 2>&1)
IP=$(echo "$LEASE" | awk '/lease of/ {print $4}')
GATEWAY=$(echo "$LEASE" | awk '/obtained from/ {print $7}' | tr -d ',')
if [ -n "$IP" ] && [ -n "$GATEWAY" ]; then
  ip addr flush dev eth0
  ip addr add "$IP/24" dev eth0
  ip route add default via "$GATEWAY"
  echo "Network configured: IP=$IP, Gateway=$GATEWAY"
else
  echo "Failed to obtain an IP address via DHCP." && exit 1
fi
\end{minted}
\end{frame}

\begin{frame}[t,fragile]
\frametitle{Networking}
\begin{center}
  \includegraphics[width=0.8\textwidth]{net.png}
\end{center}
\end{frame}

\begin{frame}[t,fragile]
\frametitle{Checkpoint: list of features}
\begin{itemize}
  \item \verb|printk| for debugging, ISA bus support, R/W block layer, compacting memory manager, PCI bus support.
  \item SATA/ATA, SCSI, E1000 (generic network card interface) support, IPv4 networking, DNS resolution, FAT32 file systems.
  \item BusyBox with a DHCP network setup script, \verb|dmesg|, \verb|wget|, \verb|vi|, \verb|sh|, \verb|awk|, \verb|dc|, \verb|bc|, \verb|httpd|, and so on.
  \item Caveat: the XZ image is 1'835'520. We will implement a custom compression algorithm to fit this in a floppy.
\end{itemize}
\end{frame}

\begin{frame}[t,fragile]
\frametitle{Kernel Compression}
  \begin{itemize}
    \item Basic idea: we put the initrd and kernel in the same logical container, which is then bootstrapped by \verb|mkpiggy| (a custom program that Linux uses for compressed kernels).
    \item Nominally, this is simple on paper: we start with some existing compression wrapper (e.g. XZ), blank out the source code, change the makefile to use a different compressor, then embed the decompressor in the kernel.
    \item Issues:
    \begin{itemize}
      \item We are looking for a somewhat fast statistical compressor that can beat XZ quite significantly.
      \item We would prefer it to not use gobs of memory and to be able to run on a 486.
      \item It needs to fit in the piggyback module together with the compressed kernel and bootloader, i.e. it can not be too large.
    \end{itemize}
  \end{itemize}
\end{frame}

\begin{frame}[t,fragile]
\frametitle{Kernel Compression}
\begin{center}
  \includegraphics[width=0.6\textwidth]{explain.png}
\end{center}
\end{frame}

\begin{frame}[t,fragile]
\frametitle{Kernel Compression}
  \begin{itemize}
    \item We will reuse some common ideas: PAQ-like context-adaptive binary arithmetic coding, a context model with a rudimentary context mixer between finite order context models and a match model that seldom disables it.
    \item The mixer output is filtered by a chain of a few APMs. The input is filtered by an E8E9 transform together with a simple disassembly pass that recognises prefixes and opcodes.
    \item The biggest issue is integrating this with the Linux kernel, which is somewhat unfamiliar to me as a C code base.
    \item If you want to know what any of this means, it will probably be explained in my compression book that I talked about here about a year ago.
  \end{itemize}
\end{frame}

\begin{frame}[t,fragile]
\frametitle{Kernel Compression}
  \begin{itemize}
    \item To not go into too much detail, the decompression stub was easily produced in a few hours.
    \item It's somewhat fast (20s to decode the kernel) on my machine.
    \item Most importantly, the kernel now fits into 1.44MB!
    \begin{Verbatim}
      1413632 arch/x86/boot/bzImage
    \end{Verbatim}
    \item Two more goal posts to score: a standalone boot loader ($\Rightarrow$ a working bootable floppy image) and some cool stuff in the user land if we can squeeze it in.
  \end{itemize}
\end{frame}

\begin{frame}[t,fragile]
\frametitle{The Bootloader}
  \begin{itemize}
    \item Written from scratch in x86 assembly, about 300 lines of code.
    \item Core difficulties are:
    \begin{itemize}
      \item Enabling the A20 gate (requires some trickery because of how many disjoint methods there are - via BIOS, via keyboard, etc).
      \item Operating in Unreal Mode so we can use BIOS interrupts to move stuff from conventional memory containing sectors read to a high memory address where the kernel expects to be loaded.
      \item Loading the GDT and properly invoking the kernel, but that's easy.
    \end{itemize}
    \item It ends up taking less than two sectors in total even with my sloppy and rushed programming.
  \end{itemize}
\end{frame}

\begin{frame}[t,fragile]
\frametitle{End Result}
  \begin{itemize}
    \item Direct download (1.44M floppy image):
    \begin{itemize}
      \item \url{https://palaiologos.rocks/doc/projects/floppy-linux.img}
    \end{itemize}
    \item This also version features \verb|fasm| (Flat Assembler by Tomek Grysztar) in the initrd.
    \item Adding an alternative program that is about 120KB in size was also possible, but it quite like \verb|fasm| and don't see the point in doing that.
  \end{itemize}
\end{frame}

\begin{frame}[t,fragile]
\frametitle{End Result: Full list of features (I)}
  \begin{itemize}
    \item \textit{All} of BusyBox; abbreviated list below:
    \begin{itemize}
      \item Shell utilities: the \verb|ash| shell, the \verb|hush| shell, \verb|base32|, \verb|base64|, \verb|chattr|, \verb|chmod|, \verb|cp|, \verb|date|, \verb|dd|, \verb|df|, \verb|egrep|, \verb|fgrep|, \verb|getopt|, \verb|iostat|, \verb|kill|, \verb|ln|, \verb|mknod|, \verb|mktemp|, \verb|mount|, \verb|more|, \verb|less|, \verb|netstat|, \verb|nice|, \verb|pipe-progress|, \verb|ps|, \verb|rm|, \verb|sync|, \verb|bc|, \verb|cal|, \verb|crc32|, \verb|dc|, \verb|diff|, \verb|du|, \verb|expand|, \verb|factor|, \verb|find|, \verb|fold|, \verb|hexdump|, \verb|install|, \verb|md5sum|, \verb|nproc|, \verb|openvt|, \verb|pgrep|, \verb|printf|, \verb|pscan|, \verb|pstree|, \verb|script|, \verb|seq|, \verb|sha*sum|, \verb|shuf|, \verb|sort|, \verb|split|, \verb|strings|, \verb|tail|, \verb|tee|, \verb|uniq|, \verb|man|.
      \item Compression/archiving: \verb|cpio|, \verb|gzip|, \verb|gunzip|, \verb|lzop|, \verb|rpm|, \verb|tar|, \verb|vi|, \verb|ar|, \verb|bunzip2|, \verb|bzip2|, \verb|dpkg|, \verb|lzcat|, \verb|lzma|, \verb|unzip|, \verb|xz|.
      \item System management: \verb|dmesg|, \verb|setfont|, \verb|powertop|, \verb|crond|, \verb|crontab|, \verb|lsof|, \verb|lspci|.
      \item Networking: \verb|ping|, \verb|udhcpc|, \verb|tftpd|, \verb|telnet|, \verb|sendmail|, \verb|ntpd|, \verb|inetd|, \verb|httpd|, \verb|ftpd|, \verb|arping|, \verb|ftpget|, \verb|ftpput|, \verb|nc|, \verb|traceroute|, \verb|wget|.
      \item Editors: \verb|ed|, \verb|sed|, \verb|awk|, \verb|vi|, \verb|hexedit|, \verb|patch|.
    \end{itemize}
  \end{itemize}
\end{frame}

\begin{frame}[t,fragile]
\frametitle{End Result: Full list of features (II)}
  \begin{itemize}
    \item Kernel: ATA, SATA, PCI, IDE support; MS-DOS filesystems (read/write), mitigations.
    \item SCSI, E1000 (generic network card interface) support, IPv4 networking, DNS resolution.
    \item Notably no IPv6 or EXT4 support (they're too bulky).
    \item Needs at least 70MB of RAM for the decompressor. Not that uncommon among i486/i586 machines.
    \item Games: \verb|2048|, \verb|chess|; Demos: \verb|donut|. 
  \end{itemize}
\end{frame}

\begin{frame}
\frametitle{Live demo!}
\end{frame}

\begin{frame}[t,fragile]
\frametitle{Hacking}
\begin{itemize}
  \item Future ideas:
  \begin{itemize}
    \item Strip down the BusyBox a bit more to make some space in the initrd for cooler stuff.
    \item Add a \verb|.profile| and a cool rice to the floppy disk Linux distribution.
  \end{itemize}
  \item Programs to bundle:
  \begin{itemize}
    \item Terminal games.
    \item Interpreters and compilers.
    \item System rescue tools?
  \end{itemize}
  \item Package manager:
  \begin{itemize}
    \item Download a platform toolchain (\url{https://skarnet.org/toolchains/native/i486-linux-musl_i486-14.2.0.tar.xz}), compile source packages.
  \end{itemize}
\end{itemize}
\end{frame}

\begin{frame}
\frametitle{Thanks!}
  \begin{itemize}
    \item \url{https://palaiologos.rocks}
    \item \url{https://github.com/kspalaiologos}
    \item \url{mailto:kszewczyk@acm.org}
    \item I hope that you liked this talk. Unfortunately due to real life circumstances and my university obligations - I am currently working on a high-profile thesis about genome compression - I started working on these slides and the code, from scratch, only on Monday 24/03 and was finished by Wednesday 26/03.
    \item This explains why the decompression performance is not top notch and the boot sector is holey.
  \end{itemize}
\end{frame}

\end{document}